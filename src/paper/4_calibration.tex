\section{Calibration}
\label{sec:calibration}

One model period corresponds to one year. Model age zero corresponds to real age 20, maximum real age is assumed to be 85, corresponding to model age 65 and retirement occurs exogenously at real 65, corresponding to model age 45.

The parameters of the production sector $[\alpha, delta_k]$ are standard values from the related literature. The parameters of the human capital formation technology, $[\delta_h, \zeta, \psi]$ are taken from \cite{LudwigSchelkleVogel2012}. The labor income tax rate throughout the entire exercise is constant at 20%.

Newborn agents enter the economy with initial assets $a_0 = 0.01$ and initial human capital level $h_0 = 1$.

As already mentioned in section \ref{sec:data}, the transition period in the model is 55 years and conditional year-to-year mortality rates are reduced by 5% relative to the previous year for years five to 29.

The parametrization of the model is summarized in table \ref{tab:calibration}.

\begin{table}[ht]
    \caption{Model calibration}
    \label{tab:calibration}
    \centering
    \begin{tabular}{l l c}
        \hline \hline
        variable    &description        &calibrated value \\
        \hline
        \csvreader[head to column names]{../../out/tables/calibration.csv}{}
        {\\\csvcolii&\csvcoliii&\csvcoliv}
        \\
        \hline \hline \\
    \end{tabular}
\end{table}