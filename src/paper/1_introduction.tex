\section{Introduction} % (fold)
\label{sec:introduction}

It is a well-known fact that the age structure in most developed nations has been changing over the past decades and will continue to change in the decades to come. These changes put the existing social security systems, especially public pension systems, under ever increasing pressure, calling for substantial reform of the systems. One aspect of demographic change, the change in old-age dependency ratios, has been documented e.g. by \cite{LudwigSchelkleVogel2012} (c.f. figure \ref{fig:OADR_original}). The prevalent changes in age structure are accompanied by changes in welfare of newborn agents entering the economy and the welfare cost of demographic change have been estimated to be substantial. Consequently, the question of how to mitigate the impact of demographic change has been subject to extensive debate. While it is quite obvious that policy adjustments are required, a necessary first step towards designing these reforms is to better understand the implications of demographic change for households and, in particular, the household's responses to these changes and potential reforms. For this, it is key to identify and quantitatively assess the mechanisms that govern household consumption, savings and labor supply decisions. While these decisions are interdependent and assessment of endogenous reactions to policy requires careful modelling of all of these, this work focuses on the households decision to acquire and maintain human capital as a component of effective labor supply.

\begin{figure}[H]
    \includegraphics[width=0.8\textwidth]{..\figures\OADR_original.png}
    \label{fig:OADR_original}
\end{figure}

The goal of this term project is to demonstrate the working of the human capital accumulation mechanism in the model by \cite{LudwigSchelkleVogel2012}. To achieve this, I focus on the relevant qualitative features of the framework, model and data developed in \cite{LudwigSchelkleVogel2012} and abstract and simplify as much as possible in all respects but the human capital accumulation mechanism. The core question to be answered in this paper is what is the households reaction in terms of human capital accumulation in light of demographic change. In this context, we model demographic change as changes in the age structure of the population, while keeping population size fixed. The basic research question is the following: Suppose there is no change in policy, yet households decide endogenously how much assets and how much human capital they want to accumulate in light of the changes in age structure documented above. How does aggregate capital and aggregate human capital / aggregate effective labor supply respond to the underlying demographic change?

The remainder of this term paper proceeds as follows. Section \ref{sec:data} briefly introduces the empirical mortality data on which the paper builds, including as the way in which demographic change is modelled and the most important qualitative features of the simulated population dynamics used for the subsequent analyses. Section \ref{sec:model} then presents the model adapted from \cite{LudwigSchelkleVogel2012} in more detail and highlights the key simplifications compared to the original model. In section \ref{sec:results}, the results of the analyses are presented and the implications for the research question are discussed. Section \ref{sec:conclusion} discusses potential extensions to the model, both with respect to the simplified version used in this paper as well as the original model used in \cite{LudwigSchelkleVogel2012} and concludes. Appendix \ref{sec:figures} contains supplementary figures and appendix \ref{sec:computations} contains additional information on the computational routine used to solve the model.
