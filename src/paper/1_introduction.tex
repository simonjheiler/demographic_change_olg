\section{Introduction} % (fold)
\label{sec:introduction}

This term paper is a replication of \cite{LudwigSchelkleVogel2012}.

Goal of this term project is to demonstrate the working of the human capital accumulation mechanism in the model by \cite{LudwigSchelkleVogel2012}. To achieve this, I focus on the relevant qualitative features of the framework, model and data developed in \cite{LudwigSchelkleVogel2012} and abstract and simplify as much as possible in all respects but the human capital accumulation mechanism. The core question to be answered in this paper is what is the households reaction in terms of human capital accumulation in light of demographic change. In this context, we model demographic change as changes in the age structure of the population, while keeping population size fixed. Age structure is modelled as a consequence of birth rates and mortality rates and demographic change is induced by changing conditional year to year mortality rates. The details regarding the modelling of population dynamics are described in more detail in section \ref{sec:data}.

The remainder of this term paper proceeds as follows. Section \ref{sec:data} briefly introduces the empirical mortality data on which the paper builds, including as the way in which demographic change is modelled and the most important qualitative features of the simulated population dynamics used for the subsequent analyses. Section \ref{sec:model} then presents the model adapted from \cite{LudwigSchelkleVogel2012} in more detail and highlights the key simplifications compared to the original model. In section \ref{sec:results}, the results of the analyses are presented and the implications for the research question are discussed. Section \ref{sec:conclusion} discusses potential extensions to the model, both with respect to the simplified version used in this paper as well as the original model used in \cite{LudwigSchelkleVogel2012} and concludes. Appendix \ref{sec:figures} contains supplementary figures and appendix \ref{sec:computations} contains additional information on the computational routine used to solve the model.
