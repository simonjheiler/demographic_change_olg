\section{Computation}
\label{sec:computation}

\subsection{Stationary equilibrium}

The solution algorithm used to solve for stationary equilibria consists of the following steps:
\begin{enumerate}
    \item Guess initial levels of aggregate capital and aggregate labor
    \item derive factor prices from firm FOCs
    \item calculate pension benefits from aggregate labor supply and the government budget constraint
    \item Solve for household value functions and policy functions by backward induction
        \begin{enumerate}
            \item Start with maximum age; value function equals flow utility from consuming all available resources
            \item Iterate backwards over ages and solve using continuation values from the next age
                \begin{enumerate}
                    \item Define meshgrids for current assets, next period assets, current human capital and next period human capital
                    \item Calculate consumption, human capital effort and flow utility on meshgrid
                    \item Extract continuation values by next period assets and next period human capital from value functions of time period $t+1$
                    \item Calculate sum of flow utility and expected discounted continuation value on meshgrids
                    \item Find maximal value over assets and human capital next period
                \end{enumerate}
            \item Store value function and policy for all ages
        \end{enumerate}
    \item Simulate cross-sectional distribution of households by asset holdings, human capital and age
        \begin{enumerate}
            \item Initiate with initial mass of household at initial levels of assets and human capital
            \item Iterate forward over policy functions to obtain cross-sectional distribution
        \end{enumerate}
    \item Aggregate over households to obtain aggregate variables
    \item Verify initial guess; If tolerance for deviation is exceeded, update guess and repeat.
\end{enumerate}

\subsection{Transitional dynamics}

The solution algorithm used to solve for transitional dynamics consists of the following steps:
\begin{enumerate}
    \item Guess initial path for aggregate capital and aggregate labor
    \item Derive factor prices from firm FOCs
    \item Calculate pension benefits from aggregate labor supply and the government budget constraint
    \item Solve for household value functions and policy functions by backward induction for all age and all time periods
        \begin{enumerate}
            \item Start with the final period $T$ and use as continuation values for all ages the value functions of the final stationary equilibrium
            \item Iterate backwards through time periods and derive value functions and optimal policies
                \begin{enumerate}
                    \item Define meshgrids for current assets, next period assets, current human capital and next period human capital
                    \item Calculate consumption, human capital effort and flow utility on meshgrid
                    \item Extract continuation values by next period assets and next period human capital from value functions of time period $t+1$
                    \item Calculate sum of flow utility and expected discounted continuation value on meshgrids
                    \item Find maximal value over assets and human capital next period
                \end{enumerate}
            \item Store policies and value functions for all ages and all time periods
        \end{enumerate}
    \item Simulate cross-sectional distribution of households by asset holdings, human capital and age over time
        \begin{enumerate}
            \item Start with cross-sectional distribution from initial stationary equilibrium
            \item Iterate forward over policy functions of the respective time period to obtain cross-sectional distribution of the following time period
        \end{enumerate}
    \item Aggregate over households to obtain paths aggregate variables
    \item Verify initial guess; If tolerance for deviation is exceeded, update guess and repeat.
\end{enumerate}

\subsection{parametrization}

The following parameters are used for computation:
\begin{itemize}
    \item Asset holdings are discretized on a linear grid with bounds $[0.001, 30.0]$ with 60 grid points
    \item Human capital is discretized on a logarithmic grid with bounds $[1.0, 5.0]$ with 20 grid points
    \item The tolerance level for deviations in aggregate capital and aggregate labor are set to $10^-5$ for computation of stationary equilibria and to $10^-3$ for the computation of transitional dynamics.
\end{itemize}