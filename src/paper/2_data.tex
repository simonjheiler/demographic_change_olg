\section{Data}
\label{sec:data}

Demographics:
The demographic parameters are chosen such that the following conditions are met:


Shifting age structure:
    initial mortality rates are taken from the Human Mortality Database (HMD). Model age zero corresponds to real age 20. I abstract from mortality prior to entering the model (implicitly setting conditional year-to-year mortality rates to zero for real ages 0 to 20). Furthermore, the maximum real age in the model is set to 90 years old, corresponding to model age 70. To achieve this, conditional year-to-year survival rates are zero for agents aged 70, irrespective of the time index.

Stationary population size: Fertility rates, i.e. the number of newborn agents each model period are chosen such that total population size remains unchanged over time


I apply an add-hoc adjustment to replicate

First, emprical survival rates are truncated at the maximum model age of

Table \ref{tab:life_expectancy} summarizes the simulated life expectancies at different points in the transition.

\begin{table}[ht]
    \caption{Simulated life expectancy}
    \label{tab:life_expectancy}
    \centering
    \begin{tabular}{l c c}
        \hline \hline
                            &at birth   & at retirement \\
        \hline
        before transition   &7.5        &1.8 \\
        at $t=20$           &88         &1.9 \\
        at $t=30$           &88         &1.9 \\
        at $t=40$           &88         &1.9 \\
        at $t=50$           &88         &1.9 \\
        at $t=60$           &88         &1.9 \\
        after transition    &8.5        &1.9 \\
        \hline \hline \\
    \end{tabular}
\end{table}
