\section{Results}
\label{sec:results}

This section summarizes the findings of the analysis. First, key qualitative features of initial and final stationary equilibrium are presented before moving to transitional dynamics between the two states.

\subsection{Stationary equilibria}

As outlined in section \ref{sec:data}, for both the initial and the final state, stationary age distributions are derived from the respective survival rates prior and post adjustment. These stationary populations and implied fertility rates are then used to solve and simulate the model. Table \ref{tab:stationary_aggregates} contains summary statistics of the two calibrations.

\begin{table}[ht]
    \caption{Stationary equilibria}
    \label{tab:stationary_aggregates}
    \centering
    \begin{tabular}{l c c c c}
        \hline \hline
            &aggregate capital  & aggregate labor   & OADR  & pension benefits \\
        \hline
        \csvreader[head to column names]{../../out/tables/stationary_aggregates.csv}{}
        {\\\csvcoli&\csvcolii&\csvcoliii&\csvcoliv&\csvcolv}
        \\
        \hline \hline \\
    \end{tabular}
\end{table}

As can be seen, the change in age structure, while keeping the population size constant, translates into an increase of the old-age dependency ratio and a decrease in pension benefits. At the same time, higher life expectancy raises the need for retirement savings, which results in a larger capital stock in the post-transition economy. In the framework presented in this paper, households do not value leisure, which means that households will invest their entire time endowment in either human capital maintenance / accumulation or labor supply. The labor supply decision and the human capital investment decision are therefore directly linked and changes in labor supply arise from investing in or maintaining higher human capital levels (given that human capital depreciation is assumed to be equal in both economies).

Figure \ref{fig:lifecycle_profiles} plots asset holdings and human capital level of the representative household over the lifecycle (assuming the household survives until the maximum age) for initial and final stationary equilibrium.

\begin{figure}[ht]
    \centering
    \figuretitle{Lifecycle profiles}
    \begin{subfigure}[b]{0.45\textwidth}
        \centering
        \includegraphics[width=\textwidth]{../../out/figures/lifecycle_profiles_initial.png}
        \caption{Initial stationary equilibrium}
        \label{fig:lifecycle_profiles_initial}
    \end{subfigure}
    \hfill
    \begin{subfigure}[b]{0.45\textwidth}
        \centering
        \includegraphics[width=\textwidth]{../../out/figures/lifecycle_profiles_final.png}
        \caption{Final stationary equilibrium}
        \label{fig:lifecycle_profiles_final}
    \end{subfigure}
    \caption{\textbf{Lifecycle profiles:} Optimal household asset holdings and human capital levels derived from policy functions}
    \label{fig:lifecycle_profiles}
\end{figure}

As is apparent from the graph, in both economies, households initially invest in human capital before accumulating capital for retirement. The rigged part of the asset profiles is an artifact of the chosen grid structure and solution method. Moving to finer grids or interpolating in between grid points for solution should remove these patterns, while leaving the overall shape of the profiles unaltered. As expected, agents in the final stationary equilibrium accumulate more of both physical and human capital in light of (i) longer expected retirement duration and (ii) lower replacement rates due to higher dependency rates.

Inequality w.r.t. to asset holdings and human capital levels is lower in the initial economy compared to the final state. The share of capital held by the richest 25\% of the population increases from 61\% in the initial stationary equilibrium to 72\% in the final state. The share of total human capital held by the bottom half of the population decreases from 32\% to 24\%. These results are not unexpected: the only source on heterogeneity in the economy is age, hence more pronounced life-cycle profiles, as featured in \ref{fig:lifecycle_profiles}, translate into more dispersion in wealth and human capital levels. Table \ref{tab:stationary_inequality} contains some summary statistics on inequality for the two stationary calibrations.

\begin{table}[ht]
    \caption{Stationary equilibria}
    \label{tab:stationary_inequality}
    \centering
    \begin{tabular}{l c c c c c c c c}
        \hline \hline
            & \multicolumn{4}{c}{Wealth} & \multicolumn{4}{c}{Human capital} \\
            &25\%  &50\% &75\% &95\% &25\%  &50\% &75\% &95\% \\
        \hline
        \csvreader[head to column names]{../../out/tables/stationary_inequality.csv}{}
        {\csvcoli&\csvcolii&\csvcoliii&\csvcoliv&\csvcolv&\csvcolvi&\csvcolvii&\csvcolviii&\csvcolix\\}
        \\
        \hline \hline \\
    \end{tabular}
\end{table}


\subsection{Transitional dynamics}

Moving to transitional dynamics, we limit our observation to the path of aggregate capital and aggregate labor over time (cf. figure \ref{fig:transition_aggregates}). Additional analyses, such as the development of inequality over time or regarding other aspects of the dynamics of the cross-sectional distribution, are, in principle, feasible with the data that is collected from solving the model. However, there is one important caveat: the transition paths resulting from the analysis are not perfectly in line with the intended model setup. As can be seen from the plot, the path for capital features a distinct and slightly concave increase at the beginning of the transition period. As discussed in section \ref{sec:data}, the intention in setting up the demographic simulation was that neither initial, nor final stationary equilibrium are directly affected by the change in mortality rates. The fact that the capital path responds strongly at the onset of the transition period is strong evidence that the time lag to the beginning of the shift in mortality rates is insufficiently long. If set up correctly, the expected transition path would remain at the initial level for some time, then smoothly adjust to the new level and again remain at that level for some time before the end of the transition period. The patterns in the results of the analysis indicate that a much longer transition period is required to obtain this. However, with the current implementation of the model, this substantially extending the transition period is not feasible due to computational limitation. As the informational content of in-depth analyses of the transition path is limited given this caveat, I abstain from further analysis at this point.